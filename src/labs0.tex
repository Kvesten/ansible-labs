\documentclass[a4paper, 12pt]{article}

\usepackage[english,russian]{babel}
\usepackage[T2A]{fontenc}
\usepackage[utf8]{inputenc}

\usepackage{indentfirst}
\usepackage{geometry}

\usepackage[normalem]{ulem}  % для зачекивания текста

\usepackage{hyperref}
\AtEndDocument{\label{lastpage}}

\geometry{left=15mm, right=15mm, top=15mm, bottom=15mm}
\setlength{\parskip}{1ex} % Отступ между абзатцами


\title{Типо лабораторка по Ansible\\№0\\Установка Ansible}
\author{Kvesten}
\date{Последние изменения: \today}

\begin{document}
    \maketitle\newpage
    \tableofcontents\newpage

    \section{Просто инфа}
        Этот пункт для умных, их ещё иногда называют задротами и батанами, так что можешь не читать, сосунок.

        \subsection{Та зачем оно мне нада?}
            И так, какие проблемы решает Ansible, и зачем он существует. Вообще, более подробно и точно можно прочесть на их сайте. Но если вкратце\footnote{Да, это правильное слово. Слова "в крации" не существует, сам в шоке\dots} то он решает вопрос автоматизации конфигурации. Если более человеческим языком, то ты говоришь ему, какую настройку и на каких компах ты хочешь видеть, а он берёт, проверяет что оно та, которая нужна, если не так то исправляет.

            Простой пример: тебе дали 1 сервер и попросили настроить его, а именно:
            \begin{enumerate}
                \item Поставить правильное имя.
                \item Настроить приём времени по сети.
                \item Поставить нужные приложения.
                \item Настроить веб-сервер.
            \end{enumerate}
            Ты скорее всего просто подключишься по ssh\footnote{Удалённый доступ к консоси.} и сделаешь всё как нужно. Хорошо. Начальство довольно, оно теперь знает что ты не лох и умеешь работать с Linux на ты, и тебе дают ещё 20 серверов и говорят, что все они должны быть так же настроены, но 1 из них должен быть балансировщиком нагрузки\footnote{То есть когда приходит запрос, он должен выбрать сервер из тех 20 и отправить на него этот запрос.}.

            Конечно ты можешь взять и подключиться к каждому по ssh и сделать всё как в пункте выше, попутно думая, что может оно не твоё? Может всё это не имеет смысла и ты всего лишь маленький незаметный винтик этой системы?\dots Конечно ты скорее всего где-то, на каком-то сервере ошибёшься, а потом будешь голову ломать с вопросам "ДА РОТ ЭТОГО СЕРВЕРА Я Е**Л, ВСЁ ТУТ ПРАВИЛЬ аааа, тут забыл рестартнуть, бывает\dots" Что бы такого не было можно описать настройку на Ansible, ему скормить сервера и ту конфигурацию, которую хочешь получить в итоге. Ansible сам сходит на них, посмотрит \sout{осудит} и настроит как надо.

        \subsection{С чем он, и как его кушать?}
            Работает Ansible на python`e, так что по идее его даже в конах\footnote{На Windows, что не понятного?} можно запустить. Но давайте будет адекватными. Хорошо? Его модно поставить в любой Linux дистрибутив в котором есть Python - то есть в любой. Как правило он доступен в пакетном менеджере вашего дистра. Для Debian/Ubuntu и производных это apt, для Fedora/кентОсь это yum, Arch/Manjaro - pacman и так далее. Если как то случилось невозможное, и его нету в репе, то можно поставить через pip\footnote{Пакетный менеджер python.}.

    \section{Задание}

        \subsection{Цель}
            Усыновить и проверить на работу Ansible. Для этого можно воспользоваться гуглом, чатГПТ, яндексом, на худой конец - мозгами\footnote{Документацией, которая есть на офф сайте.}.

        \subsection{Выполнение}
            Нужно спиратить Ansible на ваш ПиСи, и запустить команду ping для всех компудахтеров что есть (можно поставить 127.0.0.1, это значит что оно запуститься на вашем компе, но убедитесь что у вас установлен и настроен sshd). Как это сделать - гугОл. =)

        \subsection{Что я жду от тебя}
            Я это проверять не буду)) А так настроенный Ansible. Но инфы в избытке в интернете, так что просто "Удачи".
\end{document}

