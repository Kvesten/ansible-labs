\documentclass[a4paper, 12pt]{article}

\usepackage[english,russian]{babel}
\usepackage[T2A]{fontenc}
\usepackage[utf8]{inputenc}

\usepackage{indentfirst}
\usepackage{geometry}

\usepackage[normalem]{ulem}  % для зачекивания текста

\usepackage{hyperref}
\AtEndDocument{\label{lastpage}}

\geometry{left=15mm, right=15mm, top=15mm, bottom=15mm}
\setlength{\parskip}{1ex} % Отступ между абзатцами


\title{Типо лабораторка по Ansible\\№1\\Смена имени хоста.}
\author{Kvesten}
\date{Последние изменения: \today}

\begin{document}
    \maketitle\newpage
    \tableofcontents\newpage

    \section{Немного инфы для интеллектуалов}
        Эту часть можно спокойно не читать. Тут немного о том, что и как происходит в Linux относительно сети и имени компа.

        \subsection{hosts?}\ref{hosts}
            Раньше, в бородатые времена, когда не было системы DNS\footnote{DNS - Domain Name System «система доменных имён».}, все имена компьютеров в локальной (а иногда и в глобальной) сети записывались в файл /etc/hosts\footnote{Это для unix подобных осей, в окнах он где то в недрах папак system32 и тд.} в очень простом формате, а именно: \newline IP name1.com name2 pc1.kub.uni\newline где IP это IP компьютера\footnote{Неожиданно, да?}.

        \subsection{hostname?}
            А в этом файлике, который находится по пути /etc/hostname, хранится просто имя компа. Оно используется приложениями для своих нужд. Например демон\footnote{Демон, от римского чего то там в мифологии, дух хранитель. А в linux это приложение которое работает в фоне.} rsyslog\footnote{Приложение для логирования всяктх событий.} берёт оттуда имя компа для того, что бы указать его в логе и отправить дальше. Так же, оно обычно доступно по сети для обращение к "самому себе". Но это не всегда так, что бы это поправить когда оно не работает, нужно указать в файле hosts\label{hosts} запись:\newline PcName 127.0.0.1\newline Где 127.0.0.1 - loopback\footnote{Лупбэк адрес - это адрес который указывает сам на себя, его нельзя никак маршрутизировать, если хотите почитайте об этом отдельною.} адрес.

    \section{Задание}
        \subsection{Цель}
            Получить готовый к удобному использованию сервер.

        \subsection{Выполнение}
            Есть компудахтер на котором стоит ось на базе GNU/Linux, например debian. Нужно что бы после выполнения твоего playbook`а этот хост\footnote{Компудахтер, только по умному. Типо ты что то знаешь.} сменил \sout{религию} имя и мог сам себя пингануть. Далее поставить программы лоя удобного использования и настройки сервера, полностью на твоё усмотрение, я обычно ставлю htop, iotop, screen. Но ты ставь что хочешь.

        \pagebreak[4]\subsection{Что можно юзать?}
            Так как это типо лаба по типо ansible, то наверное можно использовать C++ для того что бы сменить имя компа. Шутка, асемблер. А если серьёзно =) то ansible. Из модулей есть ansible.builtin.apt и ansible.builtin.template. Я думаю сам погуглишь что и как, окей?

        \subsection{Чего я жду}
            Настроенный ansible плейбук с той задачей что была дана. Если хочешь понтануться то сделай ещё ansible.cfg и group\_vars, но это как хочешь.
\end{document}
